% Options for packages loaded elsewhere
\PassOptionsToPackage{unicode}{hyperref}
\PassOptionsToPackage{hyphens}{url}
%
\documentclass[
]{article}
\usepackage{amsmath,amssymb}
\usepackage{lmodern}
\usepackage{ifxetex,ifluatex}
\ifnum 0\ifxetex 1\fi\ifluatex 1\fi=0 % if pdftex
  \usepackage[T1]{fontenc}
  \usepackage[utf8]{inputenc}
  \usepackage{textcomp} % provide euro and other symbols
\else % if luatex or xetex
  \usepackage{unicode-math}
  \defaultfontfeatures{Scale=MatchLowercase}
  \defaultfontfeatures[\rmfamily]{Ligatures=TeX,Scale=1}
\fi
% Use upquote if available, for straight quotes in verbatim environments
\IfFileExists{upquote.sty}{\usepackage{upquote}}{}
\IfFileExists{microtype.sty}{% use microtype if available
  \usepackage[]{microtype}
  \UseMicrotypeSet[protrusion]{basicmath} % disable protrusion for tt fonts
}{}
\makeatletter
\@ifundefined{KOMAClassName}{% if non-KOMA class
  \IfFileExists{parskip.sty}{%
    \usepackage{parskip}
  }{% else
    \setlength{\parindent}{0pt}
    \setlength{\parskip}{6pt plus 2pt minus 1pt}}
}{% if KOMA class
  \KOMAoptions{parskip=half}}
\makeatother
\usepackage{xcolor}
\IfFileExists{xurl.sty}{\usepackage{xurl}}{} % add URL line breaks if available
\IfFileExists{bookmark.sty}{\usepackage{bookmark}}{\usepackage{hyperref}}
\hypersetup{
  pdftitle={Replication of `Data quality of platforms and panels for online behavioral research' by Peer et al.~(2021, Behavior Research Methods)},
  pdfauthor={Kimia Saadatian (kimia@stanford.edu)},
  hidelinks,
  pdfcreator={LaTeX via pandoc}}
\urlstyle{same} % disable monospaced font for URLs
\usepackage[margin=1in]{geometry}
\usepackage{graphicx}
\makeatletter
\def\maxwidth{\ifdim\Gin@nat@width>\linewidth\linewidth\else\Gin@nat@width\fi}
\def\maxheight{\ifdim\Gin@nat@height>\textheight\textheight\else\Gin@nat@height\fi}
\makeatother
% Scale images if necessary, so that they will not overflow the page
% margins by default, and it is still possible to overwrite the defaults
% using explicit options in \includegraphics[width, height, ...]{}
\setkeys{Gin}{width=\maxwidth,height=\maxheight,keepaspectratio}
% Set default figure placement to htbp
\makeatletter
\def\fps@figure{htbp}
\makeatother
\setlength{\emergencystretch}{3em} % prevent overfull lines
\providecommand{\tightlist}{%
  \setlength{\itemsep}{0pt}\setlength{\parskip}{0pt}}
\setcounter{secnumdepth}{-\maxdimen} % remove section numbering
\ifluatex
  \usepackage{selnolig}  % disable illegal ligatures
\fi

\title{Replication of `Data quality of platforms and panels for online
behavioral research' by Peer et al.~(2021, Behavior Research Methods)}
\author{Kimia Saadatian
(\href{mailto:kimia@stanford.edu}{\nolinkurl{kimia@stanford.edu}})}
\date{November 18, 2021}

\begin{document}
\maketitle

{
\setcounter{tocdepth}{3}
\tableofcontents
}
\hypertarget{introduction}{%
\subsection{Introduction}\label{introduction}}

\hypertarget{justification-for-choice-of-study}{%
\subsubsection{Justification for choice of
study}\label{justification-for-choice-of-study}}

I will be replicating a study by Peer et al.~(2021), which compared data
quality (attention, comprehension, reliability and honesty) between
multiple sites used for online participant recruitment: Amazon MTurk,
CloudResearch (formerly TurkPrime) and Prolific.

\hypertarget{anticipated-challenges}{%
\subsubsection{Anticipated challenges}\label{anticipated-challenges}}

As of right now, I do not anticipate running into any challenges in
replicating the results from the original study.

\hypertarget{links}{%
\subsubsection{Links}\label{links}}

Project repository: \url{https://github.com/psych251/saadatian2021.git}

Survey Preview
Link:\url{https://ucbpsych.qualtrics.com/jfe/preview/SV_cSKr5voz0akZ7P8?Q_CHL=preview\&Q_SurveyVersionID=current}

Original paper:
\url{https://github.com/psych251/saadatian2021/blob/3c2eeb3b8a352c51dea439f7878b2ab95f67acff/original-paper/Peer\%20et\%20al.\%20(2021).pdf}

Original study's OSF: \url{https://osf.io/342dp/}

\hypertarget{methods}{%
\subsection{Methods}\label{methods}}

\hypertarget{power-analysis}{%
\subsubsection{Power Analysis}\label{power-analysis}}

Main analysis I am replicating is "We found statistically significant
differences between the sites on {[}overall data quality score{]}, F(2,
1458) = 129.4, p \textless{} .001, which showed higher scores for
Prolific and CR (M = 5.87, 5.78, SD = 1.0, 1.1, respectively) compared
to MTurk (M = 4.55, SD = 1.9)).

For a power of 0.95, I will need 43 Participants in each group. However,
to get precise estimates of the differences, I will recruit 100
participants from each of the three platforms (Mturk, CloudResearch, and
Prolific), adding to a total of 300 participants.

\hypertarget{planned-sample}{%
\subsubsection{Planned Sample}\label{planned-sample}}

I will recruit participants on Amazon MTurk, CloudResearch, and
Prolific. Participants must be U.S. residents age 18 or above. Similar
to the original study, I will exclude participants who do not complete
all of the study.

``We recruited 500 participants from each platform (MTurk, CR, and
Prolific), who reported residing in the United States, in March 2021.
Participants were paid 1.5 USD on CR and MTurk and 1.1 GBP on Prolific
plus a bonus of up to 0.5 USD/GBP. We applied data quality prescreening
filters on all sites by restricting the study to participants with at
least 95\% approval rating and at least 100 previous submissions; on CR
we also used the site setting to ``block low data quality workers.'' We
excluded participants on Prolific who completed the previous study on
Prolific and participants on CR who completed the previous study on CR
or MTurk. However, because our study had to be posted twice on MTurk
(once through our MTurk account and once through our CR account), 39
participants completed the study twice, and we removed their later
submission (although they were still paid for their submissions). The
final sample thus included 1461 participants who completed the study.
Table 4 presents the samples. Additional demographics can be found in
the Appendix."

\hypertarget{materials}{%
\subsubsection{Materials}\label{materials}}

``1. Attention is measured using two attention-check questions. The
first asks participants to answer''six" and ``three'' to two items
regardless of their actual preference (other responses are coded as
failures); the second is an item within a scale worded ``I currently
don't pay attention to the questions I'm being asked in the survey''
(response other than ``strongly disagree'' is coded as a failure) 2.
Comprehension is examined through the oral summaries of instructions to
two tasks. The first task asks participants to identify faces in a
picture, but includes an instruction to only report zero; the second
includes instructions for the ``Matrix task'' (Mazar et al., 2008). Two
raters will code the oral responses independently and blind to the
origin of the participant. We will consider an answer correct if both
readers agree it is correct, and will apply a third reader to any answer
with splits. 3. Reliability will be measured using Cronbach's alpha
measure for the Need for Cognition scale. 4. Honesty will be measured
using an online version of the Matrix task (Mazar et al., 2008) that
will include two unsolvable matrices. Reporting solving any of these two
problems will be coded as a dishonest response. Additionally, we will
examine whether participants lie about their eligibility for a future
study by asking them to indicate if they want to be invited to a study
that samples participants of their own gender but whose age will be
described as 5-10 years above the age participants reported in the
beginning of the study.

In addition we will examine drop-out rates, duration for completing the
survey, overall response time and speed between sites, differences in
NFC, demographics, and patterns of usage of the site (main purpose,
frequency of usage, number of submissions and approval ratings, usage of
other sites), and we will also ask participants to report did they
complete a study similar to this study in the last months."

\hypertarget{procedure}{%
\subsubsection{Procedure}\label{procedure}}

``Participants were invited to complete a survey on individual
differences in personal attitudes, opinions, and behaviors. All
participants began the survey by answering demographic questions,
followed by the data quality measures described below. Participants
finished the survey by answering questions related to their usage of the
online platform including how often they use the site, for what
purposes, how much they earn in an average week, their percent of
approved submissions (responses that participants submit and are
approved by the researcher), and how often (if at all) they use other
sites.''

``We will apply data quality filters on all three sites in the following
scheme: 1) MTurk - only workers who have completed 100 submissions or
more and 95\% of those submissions were approved. 2) CloudResearch -
only workers who have completed 100 submissions or more and 95\% of
those submissions were approved + the default setting on CloudResearch
of''block low data quality workers"

I will be replicating their study 2, which originally averaged around
9.8 minutes long (SD = 5.2). Participants will be paid \$1.50 to
participate in this \textasciitilde10 minutes-long survey.

\hypertarget{analysis-plan}{%
\subsubsection{Analysis Plan}\label{analysis-plan}}

I plan to follow the same analyses that were conducted in the original
study.

The main analysis I am replicating is from study 2, where Peer et
al.~(2021) found higher data quality scores for data obtained from
Prolific and CloudResearch (M = 5.87, 5.78, SD = 1.0, 1.1, respectively)
compared to MTurk (M = 4.55, SD = 1.9; F(2, 1458) = 129.4, p \textless{}
.001).

In order to compute the overall data quality score, the original authors
``used chi-square tests to examine differences in the rates of
attention, comprehension and dishonesty. They also computed average
scores for (across the items per aspect) and compared them between sites
using ANOVA and regression analyses. They tested for differences between
reliability coefficients using Hakistan \& Whalen (1976) method. Lastly,
they computed an overall composite score of data quality (based on the
average scores of attention, comprehension and dishonesty) and compare
them between sites using ANOVA and regression.''

The composite Data Quality score ranged from 0 to 7 (M = 5.41; SD = 1;
Med = 6) Overall data quality composite scores ranked from highest to
lowest were: Prolific (M = 5.87, SD = 1.0) CR ( M = 5.78, SD = 1.1)
MTurk (M = 4.55, SD = 1.9)

They found statistically significant differences between the sites on
Overall Data Quality Score, F(2, 1458) = 129.4, p \textless{} .001

Post hoc tests with Bonferroni correction:

differences between Prolific and MTurk p \textless{} .001 differences
between CR and MTurk p \textless{} .001 difference between CR and
Prolific p = 0.91.

\hypertarget{differences-from-original-study}{%
\subsubsection{Differences from Original
Study}\label{differences-from-original-study}}

I do not anticipate any deviations from the original analysis plan at
this stage.

\hypertarget{methods-addendum-post-data-collection}{%
\subsubsection{Methods Addendum (Post Data
Collection)}\label{methods-addendum-post-data-collection}}

\hypertarget{actual-sample}{%
\paragraph{Actual Sample}\label{actual-sample}}

\hypertarget{differences-from-pre-data-collection-methods-plan}{%
\paragraph{Differences from pre-data collection methods
plan}\label{differences-from-pre-data-collection-methods-plan}}

\hypertarget{results}{%
\subsection{Results}\label{results}}

\hypertarget{data-preparation}{%
\subsubsection{Data preparation}\label{data-preparation}}

Data preparation following the analysis plan.

\hypertarget{confirmatory-analysis}{%
\subsubsection{Confirmatory analysis}\label{confirmatory-analysis}}

The analyses as specified in the analysis plan.

I will report differences between sites on each data quality measure and
then aggregate those findings to a composite score of data quality,
reporting differences across all sites.

ATTENTION (originally : ACQs, χ2(4) = 548.48, 203.56, p \textless{}
.001. )

COMPREHENSION (originally : χ2(4) = 152.4, p \textless{} .001 )

HONESTY (originally : χ2(4) = 153.44, p \textless{} .001 )

RELIABILITY (originally : )

OVERALL DATA QUALITY SCORES (originally: ``The score gave participants a
value between 0 and 5, showing whether they passed one or both ACQs,
answered correctly one or two comprehension questions, and did not claim
to have solved the unsolvable problem. The correlations between the five
measures ranged between 0.16 and 0.43, all p \textless{} .01, but the
overall composite score should not be considered as measuring the same
construct. Rather, it is used here as a multifacto- rial measure that
attests to the overall general level of data quality'')

\emph{Side-by-side graph with original graph is ideal here}

\begin{center}\rule{0.5\linewidth}{0.5pt}\end{center}

\hypertarget{exploratory-analyses}{%
\subsubsection{Exploratory analyses}\label{exploratory-analyses}}

\hypertarget{discussion}{%
\subsection{Discussion}\label{discussion}}

\hypertarget{summary-of-replication-attempt}{%
\subsubsection{Summary of Replication
Attempt}\label{summary-of-replication-attempt}}

-\textgreater{} Open the discussion section with a paragraph summarizing
the primary result from the confirmatory analysis and the assessment of
whether it replicated, partially replicated, or failed to replicate the
original result.

\hypertarget{commentary}{%
\subsubsection{Commentary}\label{commentary}}

-\textgreater{} dd open-ended commentary (if any) reflecting (a)
insights from follow-up exploratory analysis, (b) assessment of the
meaning of the replication (or not) - e.g., for a failure to replicate,
are the differences between original and present study ones that
definitely, plausibly, or are unlikely to have been moderators of the
result, and (c) discussion of any objections or challenges raised by the
current and original authors about the replication attempt. None of
these need to be long.

\end{document}
